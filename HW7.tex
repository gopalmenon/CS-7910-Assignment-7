%%%%%%%%%%%%%%%%%%%%%%%%%%%%%%%%%%%%%%%%%
% Short Sectioned Assignment
% LaTeX Template
% Version 1.0 (5/5/12)
%
% This template has been downloaded from:
% http://www.LaTeXTemplates.com
%
% Original author:
% Frits Wenneker (http://www.howtotex.com)
%
% License:
% CC BY-NC-SA 3.0 (http://creativecommons.org/licenses/by-nc-sa/3.0/)
%
%%%%%%%%%%%%%%%%%%%%%%%%%%%%%%%%%%%%%%%%%

%----------------------------------------------------------------------------------------
%	PACKAGES AND OTHER DOCUMENT CONFIGURATIONS
%----------------------------------------------------------------------------------------

\documentclass[paper=a4, fontsize=11pt]{scrartcl} % A4 paper and 11pt font size

\usepackage[T1]{fontenc} % Use 8-bit encoding that has 256 glyphs
\usepackage{fourier} % Use the Adobe Utopia font for the document - comment this line to return to the LaTeX default
\usepackage[english]{babel} % English language/hyphenation
\usepackage{amsmath,amsfonts,amsthm} % Math packages

\usepackage{sectsty} % Allows customizing section commands
\usepackage[top=5em]{geometry}
\allsectionsfont{\centering \normalfont\scshape} % Make all sections centered, the default font and small caps

\usepackage{fancyhdr} % Custom headers and footers
\pagestyle{fancyplain} % Makes all pages in the document conform to the custom headers and footers
\fancyhead{} % No page header - if you want one, create it in the same way as the footers below
\fancyfoot[L]{} % Empty left footer
\fancyfoot[C]{} % Empty center footer
\fancyfoot[R]{\thepage} % Page numbering for right footer
\renewcommand{\headrulewidth}{0pt} % Remove header underlines
\renewcommand{\footrulewidth}{0pt} % Remove footer underlines
\setlength{\headheight}{5pt} % Customize the height of the header

\numberwithin{equation}{section} % Number equations within sections (i.e. 1.1, 1.2, 2.1, 2.2 instead of 1, 2, 3, 4)
\numberwithin{figure}{section} % Number figures within sections (i.e. 1.1, 1.2, 2.1, 2.2 instead of 1, 2, 3, 4)
\numberwithin{table}{section} % Number tables within sections (i.e. 1.1, 1.2, 2.1, 2.2 instead of 1, 2, 3, 4)

\setlength\parindent{0pt} % Removes all indentation from paragraphs - comment this line for an assignment with lots of text

\usepackage{mathtools}
\usepackage{amssymb}
\usepackage{gensymb}
\usepackage{chngcntr}
\usepackage{csquotes}
\usepackage{flexisym}
\usepackage{algorithm,algpseudocode}
\usepackage{tikz}

\usepackage{verbatim}
\usetikzlibrary{arrows,shapes}

\newcommand\Mycomb[2][n]{\prescript{#1\mkern-0.5mu}{}C_{#2}}

\counterwithout{figure}{section}
%----------------------------------------------------------------------------------------
%	TITLE SECTION
%----------------------------------------------------------------------------------------

\newcommand{\horrule}[1]{\rule{\linewidth}{#1}} % Create horizontal rule command with 1 argument of height

\title{	
\normalfont \normalsize 
\textsc{Utah State University, Computer Science Department} \\ [25pt] % Your university, school and/or department name(s)
\horrule{0.5pt} \\[0.4cm] % Thin top horizontal rule
\huge CS 7910 Computational Complexity\\Assignment 7 \\ % The assignment title
\horrule{2pt} \\[0.5cm] % Thick bottom horizontal rule
}

\author{Gopal Menon} % Your name

\date{\normalsize\today} % Today's date or a custom date

\begin{document}

\maketitle % Print the title

\begin{enumerate}
\item \textbf{(10 points)} We have studied a 2-approximation algorithm for the vertex cover problem in class. Give an example of a graph such that the size of the vertex cover computed by our algorithm is always equal to $2 * OPT$, where $OPT$ is the size of the vertex cover in an optimal solution.

This actually proves that the approximation ratio 2 is \enquote{tight} for our algorithm.

\pgfdeclarelayer{background}
\pgfsetlayers{background,main}

\begin{frame}

\tikzstyle{vertex}=[circle,fill=black!25,minimum size=20pt,inner sep=0pt]
\tikzstyle{selected vertex} = [vertex, fill=red!24]
\tikzstyle{edge} = [draw,thick,-]
\tikzstyle{weight} = [font=\small]
\tikzstyle{selected edge} = [draw,line width=5pt,-,red!50]
\tikzstyle{ignored edge} = [draw,line width=5pt,-,black!20]

\begin{figure}[hb]
\begin{tikzpicture}[scale=2, auto,swap]
    % Draw a 7,11 network
    % First we draw the vertices
    \foreach \pos/\name in {{(0,0)/a}, {(2,0)/b}, {(1.4142,1.4142)/c},
                            {(0,2)/d},{(-1.4142,1.4142)/e},{(-2,0)/f}, {(-1.4142, -1.4142)}/g, {(0,-2)/h},{(1.4142,-1.4142)/i}}
        \node[vertex] (\name) at \pos {$\name$};
    % Connect vertices with edges and draw weights
    \foreach \source/ \dest /\weight in {a/b/, a/c/,a/d/,a/e/,
                                         a/f/, a/g/,a/h/,a/i/}
        \path[edge] (\source) -- node[weight] {$\weight$} (\dest);
    % Start animating the vertex and edge selection. 


\end{tikzpicture}
\caption{An example of a star shaped graph}
\label{StarGraph}
\end{figure}


\end{frame}

Figure \ref{StarGraph} shows an example of a star shaped graph where node $a$ is connected to every other node by an edge. In such a graph, the set of vertices with just the node $a$ will be the vertex cover since the node will cover all the edges. If we use the 2-approximation algorithm, the vertex cover will include the two vertices in a randomly selected edge. So in such a graph, the 2-approximation algorithm will always return a vertex set of size $2$, which will be twice the optimum solution of size $1$. 
\end{enumerate}

%----------------------------------------------------------------------------------------

\end{document}