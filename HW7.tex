%%%%%%%%%%%%%%%%%%%%%%%%%%%%%%%%%%%%%%%%%
% Short Sectioned Assignment
% LaTeX Template
% Version 1.0 (5/5/12)
%
% This template has been downloaded from:
% http://www.LaTeXTemplates.com
%
% Original author:
% Frits Wenneker (http://www.howtotex.com)
%
% License:
% CC BY-NC-SA 3.0 (http://creativecommons.org/licenses/by-nc-sa/3.0/)
%
%%%%%%%%%%%%%%%%%%%%%%%%%%%%%%%%%%%%%%%%%

%----------------------------------------------------------------------------------------
%	PACKAGES AND OTHER DOCUMENT CONFIGURATIONS
%----------------------------------------------------------------------------------------

\documentclass[paper=a4, fontsize=11pt]{scrartcl} % A4 paper and 11pt font size

\usepackage[T1]{fontenc} % Use 8-bit encoding that has 256 glyphs
\usepackage{fourier} % Use the Adobe Utopia font for the document - comment this line to return to the LaTeX default
\usepackage[english]{babel} % English language/hyphenation
\usepackage{amsmath,amsfonts,amsthm} % Math packages

\usepackage{sectsty} % Allows customizing section commands
\usepackage[top=5em]{geometry}
\allsectionsfont{\centering \normalfont\scshape} % Make all sections centered, the default font and small caps

\usepackage{fancyhdr} % Custom headers and footers
\pagestyle{fancyplain} % Makes all pages in the document conform to the custom headers and footers
\fancyhead{} % No page header - if you want one, create it in the same way as the footers below
\fancyfoot[L]{} % Empty left footer
\fancyfoot[C]{} % Empty center footer
\fancyfoot[R]{\thepage} % Page numbering for right footer
\renewcommand{\headrulewidth}{0pt} % Remove header underlines
\renewcommand{\footrulewidth}{0pt} % Remove footer underlines
\setlength{\headheight}{5pt} % Customize the height of the header

\numberwithin{figure}{section} % Number figures within sections (i.e. 1.1, 1.2, 2.1, 2.2 instead of 1, 2, 3, 4)
\numberwithin{table}{section} % Number tables within sections (i.e. 1.1, 1.2, 2.1, 2.2 instead of 1, 2, 3, 4)

\setlength\parindent{0pt} % Removes all indentation from paragraphs - comment this line for an assignment with lots of text

\usepackage{mathtools}
\usepackage{amssymb}
\usepackage{gensymb}
\usepackage{chngcntr}
\usepackage{csquotes}
\usepackage{flexisym}
\usepackage{algorithm,algpseudocode}
\usepackage{tikz}

\usepackage{verbatim}
\usetikzlibrary{arrows,shapes}

\newcommand\Mycomb[2][n]{\prescript{#1\mkern-0.5mu}{}C_{#2}}

\counterwithout{figure}{section}
%----------------------------------------------------------------------------------------
%	TITLE SECTION
%----------------------------------------------------------------------------------------

\newcommand{\horrule}[1]{\rule{\linewidth}{#1}} % Create horizontal rule command with 1 argument of height

\title{	
\normalfont \normalsize 
\textsc{Utah State University, Computer Science Department} \\ [25pt] % Your university, school and/or department name(s)
\horrule{0.5pt} \\[0.4cm] % Thin top horizontal rule
\huge CS 7910 Computational Complexity\\Assignment 7 \\ % The assignment title
\horrule{2pt} \\[0.5cm] % Thick bottom horizontal rule
}

\author{Gopal Menon} % Your name

\date{\normalsize\today} % Today's date or a custom date

\begin{document}

\maketitle % Print the title

\begin{enumerate}
\item \textbf{(10 points)} We have studied a 2-approximation algorithm for the vertex cover problem in class. Give an example of a graph such that the size of the vertex cover computed by our algorithm is always equal to $2 * OPT$, where $OPT$ is the size of the vertex cover in an optimal solution.

This actually proves that the approximation ratio 2 is \enquote{tight} for our algorithm.
\begin{comment}

\end{comment}
\pgfdeclarelayer{background}
\pgfsetlayers{background,main}

\begin{frame}

\tikzstyle{vertex}=[circle,fill=black!25,minimum size=20pt,inner sep=0pt]
\tikzstyle{selected vertex} = [vertex, fill=red!24]
\tikzstyle{edge} = [draw,thick,-]
\tikzstyle{weight} = [font=\small]
\tikzstyle{selected edge} = [draw,line width=5pt,-,red!50]
\tikzstyle{ignored edge} = [draw,line width=5pt,-,black!20]

\begin{figure}[hb]
\centering
\begin{tikzpicture}[scale=2, auto,swap]
    % Draw a 7,11 network
    % First we draw the vertices
    \foreach \pos/\name in {{(0,0)/a}, {(2,0)/b}, {(1.4142,1.4142)/c},
                            {(0,2)/d},{(-1.4142,1.4142)/e},{(-2,0)/f}, {(-1.4142, -1.4142)}/g, {(0,-2)/h},{(1.4142,-1.4142)/i}}
        \node[vertex] (\name) at \pos {$\name$};
    % Connect vertices with edges and draw weights
    \foreach \source/ \dest /\weight in {a/b/, a/c/,a/d/,a/e/,
                                         a/f/, a/g/,a/h/,a/i/}
        \path[edge] (\source) -- node[weight] {$\weight$} (\dest);
    % Start animating the vertex and edge selection. 


\end{tikzpicture}
\caption{An example of a star shaped graph}
\label{StarGraph}
\end{figure}


\end{frame}

Figure \ref{StarGraph} shows an example of a star shaped graph where node $a$ is connected to every other node by an edge. In such a graph, the set of vertices with just the node $a$ will be the vertex cover since the node will cover all the edges. If we use the 2-approximation algorithm, the vertex cover will include the two vertices in a randomly selected edge. So in such a graph, the 2-approximation algorithm will always return a vertex set of size $2$, which will be twice the optimum solution of size $1$. 

\item \textbf{(10 points)} Let $G$ be an undirected graph of $n$ vertices and $m$ edges. Let $V$ denote the vertex set of $G$. Recall that we have proved the following observation about the relationship between a vertex cover and an independent set.

\textbf{Observation 1} A subset $S \subseteq V$ is a vertex cover of $G$ if and only if the subset $S\textprime = V - S$ is an independent set of $G$.

In class we have designed a 2-approximation algorithm for computing a vertex cover. Suppose we use the following approximation algorithm to compute an independent set of $G$.

First, we use the algorithm studied in class to compute a vertex cover $S$ of $G$. Then, we return $S\textprime = V - S$ as the solution for our independent set problem. By the above observation, $S\textprime$ is an independent set of $G$.

Since the approximation ratio of the algorithm for computing the vertex cover is 2, is the approximation ratio of the above algorithm for computing an independent set also 2? 
Please justify your answer.\\

Let $VC_2$ and $IS_2$ be the Vertex Cover and Independent Set found by the 2-approximation algorithm as described above. Let the corresponding optimal solutions be denoted by $VC_{OPT}$ and $IS_{OPT}$.

We know the following are true
\begin{align*} \label{ISplusVC}
&\left | VC_2 \right | + \left | IS_2 \right | = \left | V \right |\\
&\left | VC_{OPT} \right | + \left | IS_{OPT} \right | = \left | V \right |\\
&\left | VC_2 \right | \leq 2 \left | VC_{OPT} \right |
\end{align*}

It follows from the above that
\begin{align*} 
\left | V \right | - \left | IS_2 \right | &\leq 2( \left | V \right | - \left| IS_{OPT} \right|)\\
2 \left | IS_{OPT} \right | &\leq 2 \left | V \right | - \left | V \right | + \left | IS_2 \right |\\
\end{align*}
\begin{equation} 
\label{eq1}
\left | IS_{OPT} \right | \leq \frac{1}{2} (\left | V \right | + \left | IS_2 \right |)
\end{equation}

For the independent set obtained by the above algorithm to have an approximation ratio of 2, the following should be true
\begin{align}
\label{eq2} 
\left | IS_{OPT} \right | \leq 2 \left | IS_2 \right |)
\end{align}

Equation \ref{eq1} can be changed as follows without affecting the inequality.
\begin{equation} 
\label{eq3}
\left | IS_{OPT} \right | \leq 2 \left | IS_2 \right | + \frac{1}{2} \left | V \right |
\end{equation}

Since equation \ref{eq3} cannot be reduced to equation \ref{eq2}, I would say that the independent set does not have an approximation ration of 2.

\item \textbf{(20 points)} In this exercise, we transform an instance of traveling salesman problem into another instance whose edge weights satisfy the triangle inequality.

Specifically, let $G$ be a complete undirected graph of $n$ vertices in which each edge $(u, v)$ has a weight $w(u, v) \geq 0$. However, the edge weights of $G$ may not satisfy the triangle inequality.

\begin{enumerate}
\item In polynomial time, construct another complete undirected graph $G\textprime$ of $n$ vertices such that the following hold: (1) each edge $(u,v)$ has a weight $w\textprime(u,v) \geq 0$; (2) the edge weights of $G\textprime$ satisfy the triangle inequality; (3) a TSP optimal cycle of $G$ corresponds to a TSP optimal cycle of $G\textprime$.

\item In class we have designed a 2-approximation algorithm for the TSP problem with the triangle inequality. Because of the above (a), does this mean that there also exists a 2-approximation algorithm for the general TSP problem without the triangle inequality? Please justify your answer.\\
\end{enumerate}

\begin{enumerate}
\item In order to construct graph $G\textprime$, make it the same as graph $G$, with the difference that every edge has a weight of 1. This construction can be done in $O(V+E)$ time or polynomial time (here $V$ and $E$ are the number of vertices and edges in the graphs). Since every edge has a weight of 1, any triangle in the graph will satisfy the triangle inequality. A TSP optimal cycle in graph $G$ will map to a cycle in graph $G\textprime$ that starts at a node, visits every other node and returns to the starting node. This cycle in graph $G\textprime$ will be a TSP optimal cycle since the length of the path will be equal to the number of nodes, and any cycle that starts at a node, visits every other path and returns to the starting node will have a length equal to the number of nodes.

\item If we can show that there exists a 2-approximation algorithm for the general TSP problem without the triangle inequality 
\begin{enumerate}
\item if and only if there exists a TSP problem with triangle inequality having a 2-approximation algorithm

\item the TSP problem with triangle inequality was constructed in polynomial time from the TSP problem without triangle inequality.
\end{enumerate}

Then we can say that there exists a 2-approximation algorithm for the general TSP problem without the triangle inequality.

We have already shown that (ii) is true. We know that the TSP problem with the triangle inequality has a 2-approximation solution. The equivalent TSP problem without the triangle inequality, will have a cycle going through the corresponding nodes. This cycle may not satisfy that condition that the $Cycle \: Weight \leq 2 * TSP \: optimal \: cycle \: weight$ because the triangle inequality may not be satisfied. So the cycle may not be limited by $2 * TSP \: optimal \: cycle \: weight$.

This means that even though (a) is true, there may not exist a 2-approximation algorithm for the general TSP problem without the triangle inequality.
\end{enumerate}




\end{enumerate}

%----------------------------------------------------------------------------------------

\end{document}